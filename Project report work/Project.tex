\documentclass[10pt,a4paper]{report}
\usepackage[utf8]{inputenc}
\usepackage{amsmath}
\usepackage{amsfonts}
\usepackage{amssymb}
\usepackage{graphicx}
\usepackage{float}
\author{Group 5:\\Mohamed Khalid, AE13B013,\\G.D.Sunil,AE13B049,\\Manoj Velmurugan,AE14B013,\\Radhakrishnan B, AE14B024,\\Mohamed Ajmal, AE14B043 }
\title{Spacecraft dynamics project: Controller design for Zero momentum biased Satellite-Oceansat.}
\begin{document}
\maketitle
\tableofcontents
\chapter{Problem Statement}
Design a preliminary attitude control system for a satellite. The satellite can have any of the following stabilisation methods given below.
\begin{enumerate}
\item Gravity gradient stabilisation
\item Spin/dual spin stabilisation with passive/active magnetic torque for damping. 
\item Momentum biased stabilisers with earth sensors measuring roll and pitch as primary sensors with gyroscopes and schemes for momentum dumping using thrusters.
\item 0 momentum biased spacecraft with star sensor for roll, pitch and yaw euler parameters with gyroscopes along with momentum dumping mechanism of wheels.\\
\\
\end{enumerate}

\emph{\textbf{Steps involved:}}
\begin{enumerate}
\item Select a suitable kinematic system for spacecraft.
\item Using Euler's equation derive dynamical equations of motion and include gravity gradient torque.
\item Study stability dynamics of the system of both pitch motion and roll-
yaw motion and figure out what kind of motion is possible. Also describe why active control system is required.
\item Design a control system accordingly to control spacecraft with PID strategies.
\item Figure out a control strategy to momentum dump with selected wheel based control.
\item Select proportional control gain accounting for maximum allowable steady state 
of 0.005 deg about all axes (for zero-momentum biased system).

\end{enumerate}

\textbf{Problem Assigned:}\\
Oceansat-1 is a 3-axis stabilized earth pointing satellite with a 4-wheel configuration which is traditional. That is the wheel configuration is with a wheel about each principal axis and the 4th wheel is mounted with 54.7 deg with respect to all three wheels. Nominally the principal axis wheels are rotated with 1000 rpm and the redundant wheel is rotated with -1732 rpm so that zero-momentum is achieved.
\begin{figure}[H]
\centering

\caption{Configuration of wheels}
\end{figure}
The momentum dumping is achieved by using 60 Am$ ^{2} $ torque rods about all the three axes. Actual MI properties of the s/c after deployment are,
\begin{equation}
J_{c}=\begin{bmatrix}
1800 & -50 & -15 \\
-50 & 1600 & 25 \\
-15 & 25 & 1200 \\
\end{bmatrix} Kg m^{2}
\end{equation}
Where the mass is given to be 1600 Kg, and [x,y,z] correspond to yaw, roll and pitch axes respectively.\\ \\
Initially assume that the cross product of inertias is negligible and design
the control system. Then when you actually apply the control, use the
actual inertia matrix and compare and comment how the performance
varies.
\\ \\
Use momentum dumping by torque rods about 2 axes and design PID control for $ T_{x}= T_{z}=2*10^{-3}Nm $ and $ T_{y}=10^{-4}Nm $ with $ \omega_{0}= 1.0741*10^{-3} rad/s $.
Also compare strategy and time responses for tetrahedron and Pyramid configurations.
\\\begin{figure}[H]
\centering
\includegraphics[scale=0.5]{image_gallery.png}
\caption{Ocean Sat}
\end{figure}
\chapter{Dynamic Model of spacecraft}
The dynamics for the spacecraft is modelled using Euler angles since the spacecraft is expected to operate within small ranges of Euler angles(Earth pointing satellite) and it is easier to design controller for euler angles.\\ \\
\textbf{Coordinate system} used(as given in the question):\\
Z axis-Pitch\\
Y axis-Roll\\
X axis-Yaw\\
Order of rotation is 3-2-1(z-y-x)\\
where,\\
yaw axis points towards Earth. Roll angle is in the direction of motion of the satellite and pitch axis as per right hand coordinate system.
\\ \\
The \textbf{Dynamics Equations} of motion for the spacecraft are given as follows.\\
\begin{equation}
I\dot{\omega}+\omega\times I(\omega)=T_{disturbance}+T_{magnetic}-T_{RW}+T_{g}
\end{equation}

\begin{equation}
\omega=\left(\omega_{b}-C_{b0}\begin{bmatrix}
0\\0\\\omega_{0}
\end{bmatrix}\right)
\end{equation}
Where,\\
	$ T_{magnetic} $ represents torque by magnetic torquers due to momentum dumping \\
	$ T_{RW} $ represents torque due to reaction wheels.\\
	$ T_{g} $ represents torque due to gravity gradient.\\
	$ \omega $ represents angular velocity of satellite wrt inertially fixed frame and written in body frame.\\
	$ \omega_b $ represents angular velocity of satellite wrt orbit and written/represented in body frame.

<<<<<<< HEAD
=======
\section{Reaction wheel Torque:}
\subsection{Traditional Configuration:}
For reaction wheels in Traditional configuration, the net torque due to the wheels come out as,
\begin{equation}
T_{RW}=\begin{bmatrix}
1& 0 &0 &\frac{-1}{\sqrt{3}}\\
1& 0 &0 &\frac{-1}{\sqrt{3}}\\
1& 0 &0 &\frac{-1}{\sqrt{3}}\\
1& 0 &0 &\frac{-1}{\sqrt{3}}\\
\end{bmatrix}
\begin{bmatrix}
T_{1}\\T_{2}\\T_{3}\\T_{4}
\end{bmatrix}
\end{equation}
Where, $ T_{i}=I_{w}\dot{\omega_{i}} $ of the $ i^{th} $ wheel, and the fourth wheel is at equal angles to the other three wheels.\\

The distribution matrix obtained by finding inverse relation using the pseudo inverse matrix, where for this case,
\begin{equation}
\begin{bmatrix}
T_{1}\\T_{2}\\T_{3}\\T_{4}
\end{bmatrix}
=\frac{1}{2}\begin{bmatrix}
\frac{5}{3} &-\frac{1}{3} &-\frac{1}{3} \\
\frac{-1}{3} &\frac{5}{3} & -\frac{1}{3}\\
\frac{-1}{3} &-\frac{1}{3} &\frac{5}{3} \\
\frac{-1}{\sqrt{3}} &\frac{-1}{\sqrt{3}} &\frac{-1}{\sqrt{3}} 
\end{bmatrix}T_{RW}
\end{equation}

%Not included
%\begin{equation}
%T_{RW}=I_{w}\begin{bmatrix}
%\dot{\omega_{1}}-\frac{\dot{\omega_{4}}}%{\sqrt{3}}\\
%\dot{\omega_{2}}-\frac{\dot{\omega_{4}}}{\sqrt{3}}\\
%\dot{\omega_{3}}-\frac{\dot{\omega_{4}}}%{\sqrt{3}}
%\end{bmatrix}
%\end{equation}
%Not to be included
%+I_{w}\begin{bmatrix}
%\omega_{3}\omega_{by}-\omega_{2}\omega_{bz}%\\
%\omega_{1}\omega_{bz}-\omega_{3}\omega_{bx}%\\
%\omega_{2}\omega_{bx}-\omega_{1}\omega_{by}
%\end{bmatrix}

Where, \\
$\omega_{i}-$angular velocity of wheels wrt to the satellite. In normal operation all $\omega_{i}$ will be positive. Sign convention is taken so. \\
$ \omega_{b}- $(angular velocity of satellite body wrt orbit) is given by,\\
>>>>>>> 382f0ed216ab024d9aa926e96a16e603d7a4de2d
$
\omega_{b} =\begin{bmatrix}
-sin(r) & 0 & 1\\
cos(r) sin(y)&cos(y)&0\\
cos(r) cos(y)&-sin(y)&0
\end{bmatrix}
$ $
\begin{bmatrix}
\dot{p}\\ \dot{r} \\ \dot{y}
\end{bmatrix}
$\\ \\
where [$\textbf{x}$,$\textbf{y}$,$\textbf{z}$]  correspond to the yaw(y), roll(r) and pitch(p) axes respectively and $ C_{b0} $ corresponds to the rotation matrix by,
\begin{equation}
C_{b0}=C_{x}(y)C_{y}(r)C_{z}(p)
\end{equation}
by the 3-2-1 rotation convention.
\section{Reaction wheel Torque:}
\subsection{Traditional Configuration:}
For reaction wheels in Traditional configuration, the net torque due to the wheels come out as,
\begin{equation}
T_{RW}=I_{w}\begin{bmatrix}
\dot{\omega_{1}}-\frac{\omega_{4}}{\sqrt{3}}\\
\dot{\omega_{2}}-\frac{\omega_{4}}{\sqrt{3}}\\
\dot{\omega_{3}}-\frac{\omega_{4}}{\sqrt{3}}
\end{bmatrix}
\end{equation}
Where, \\
$\omega_{i}-$angular velocity of wheels wrt to the satellite. In normal operation all $\omega_{i}$ will be positive. Sign convention is taken so. \\

\newpage
\subsection{Tetrahedron Configuration:}
<<<<<<< HEAD
%\section{Gravity Gradient Torque:}
%The gravity gradient torque is given by,
%	\begin{equation}
%	T_{g}=\frac{3\mu}{r^{5}}r_{b}^{\times}Ir_{b}=3\omega_{0}^{2}\begin{bmatrix}
%	(I_{z}-I_{y})y\\(I_{z}-I_{x})r \\0
%	\end{bmatrix}
%	\end{equation}
%	Where $ \omega_{0} =\sqrt{\frac{\mu}{R_{0}^{3}}}$.\\
%This torque is only used in the final simulation of satellite with the controller and not used in the design of controller.

=======

For reaction wheels in the tetrahedral configuration, 
\begin{figure}[H]
\centering
\includegraphics[scale=0.5]{Tetconfig.png}
\caption{Zero momentum biased Tetrahedral configuration.}
\end{figure}

Using the coordinates of each wheel, we get,
\begin{equation}
T_{RW}=\begin{bmatrix}
\frac{1}{\sqrt{3}} &\frac{-1}{\sqrt{3}}&\frac{-1}{\sqrt{3}}&\frac{1}{\sqrt{3}} \\
\frac{1}{\sqrt{3}} &\frac{-1}{\sqrt{3}}&\frac{1}{\sqrt{3}}&\frac{-1}{\sqrt{3}} \\
\frac{1}{\sqrt{3}} &\frac{1}{\sqrt{3}}&\frac{-1}{\sqrt{3}}&\frac{-1}{\sqrt{3}} 
\end{bmatrix}
\begin{bmatrix}
T_{1}\\T_{2}\\T_{3}\\T_{4}
\end{bmatrix}
\end{equation}
Where, $ T_{i}=I_{w}\dot{\omega_{i}} $ of the $ i^{th} $ wheel.
\\The distribution matrix obtained by finding inverse relation using the pseudo inverse matrix.
\begin{equation}
\begin{bmatrix}
T_{1}\\T_{2}\\T_{3}\\T_{4}
\end{bmatrix}
=\frac{1}{4}\begin{bmatrix}
\sqrt{3}&\sqrt{3}&\sqrt{3}\\
-\sqrt{3}&-\sqrt{3}&\sqrt{3}\\
-\sqrt{3}&\sqrt{3}&-\sqrt{3}\\
-\sqrt{3}&-\sqrt{3}&\sqrt{3}\\
\end{bmatrix}
T_{RW}
\end{equation}

\section{Gravity Gradient Torque:}
The gravity gradient torque is given by,
	\begin{equation}
	T_{g}=\frac{3\mu}{r^{5}}r_{b}^{\times}Ir_{b}=3\omega_{0}^{2}\begin{bmatrix}
	(I_{z}-I_{y})y\\(I_{z}-I_{x})r \\0
	\end{bmatrix}
	\end{equation}
	Where $ \omega_{0} =\sqrt{\frac{\mu}{R_{0}^{3}}}$.\\
This torque is only used in the final simulation of satellite with the controller and not used in the design of controller.
>>>>>>> 382f0ed216ab024d9aa926e96a16e603d7a4de2d
\section{Magnetic Torque}
%The magnetic torque is based on a magnetic dipole system model of the earth where the satellite is in orbit about. Hence, for any point in the orbit, the magnetic field around the satellite can be easily found using the model and assuming that the dipole is aligned at angle of 11$ ^{0} $ wrt earth's true axis, we find the radial and azimuthal fields to be,
%\begin{equation}
%B_{r}=-2B_{0}\left(\frac{R_{E}}{r}\right)^{3}cos(\theta)
%\end{equation}
%\begin{equation}
%B_{\theta}=-B_{0}\left(\frac{R_{E}}{r}\right)^{3}sin(\theta)
%\end{equation}
%\begin{equation}
%|B|=B_{0}\left(\frac{R_{E}}{r}\right)^{3}\sqrt{1+3cos^{2}(\theta)}
%\end{equation}
%where, $ R_{E} $ is the mean radius of the earth, $ \theta $ is the azimuth from the north magnetic pole,r is the distance from earth's center and $B_{0}=3.12\times10^{-5}T$. There is no need to consider solar wind effects since the satellite OCEANSAT revolves closely about earth.

Magnetic torque is given by,
\begin{equation}
\tau=M X B
\end{equation}
where,\\ \\
B- magnetic field of Earth Represented in body frame. This is obtained using DMSP magnetic model(matlab code given in http://www.geomag.us/models/dmsp.html)\\ 
M- magnetic dipole moment of 3 coils.
Coils are assumed to be PWM controlled. So the duty cycle is going to control the resulting magnetic torque on the satellite \\
M= [dutyCycle1*m; dutyCycle2*m; dutyCycle3*m]\\
m= $60A/m^2$


\chapter{Stability analysis}
Deriving the equations of motion, we get (The equations of motion remain same, but [$I_{x},I_{y},I_{z}$] of new frame become [$I_{y},I_{z},I_{x}$] of old frame.\\
Hence, the equations for stability come out to be,
\begin{equation}
I_{y}\ddot{r}-[I_{y}-I_{z}+I_{x}]\omega_{0}\dot{y}+4\omega_{0}^{2}(I_{z}-I_{x})r=0
\end{equation}
\begin{equation}
I_{z}\ddot{p}+3\omega_{0}^{2}(I_{y}-I_{x})p=0
\end{equation}
\begin{equation}
I_{x}\ddot{y}-[I_{y}-I_{z}+I_{x}]\omega_{0}\dot{r}+\omega_{0}^{2}(I_{z}-I_{x})r=0
\end{equation}

\section{Stability of pitch motion}
For stability of pitch, which has the form,
\begin{equation}
\ddot{p}+\alpha p=0
\end{equation}
$ \alpha>0 $ and hence $ I_{y}>I_{x} $. This is not true and hence the satellite is pitch unstable.
\section{Roll and yaw unstability}
In similar fashion, we see that if $ I_{z}>I_{x},I_{y} $, roll yaw is stable. However, by given info, the roll yaw is unstable. Hence, the satellite is gravity gradient unstable. 
\section{Need for active control}
The active control system is needed because of the unstable roll yaw motion and pitch motion. Because of constant disturbance torque given in the question. Dumping of Momentum gained will be required. This is done using 3 axis torque rods on satellite
\chapter{Controller design:}
\section{Simplified equations}
Linearised Satellite equations are as follows
\[
Ixx\ddot{(yaw)}=Tdx-Tcx
\]

\[
Iyy\ddot{(roll)}=Tdy-Tcy
\]

\[
Izz\ddot{(pitch)}=Tdz-Tcz
\]

Gravity gradient torque is not included here because it is already
included in the disturbance torque for design. 
\[
Tcx=Kp*e+Kd*\dfrac{de}{dt}+Ki*\int edt
\]

Converting the equations to Transfer function form, we get,

\[
e(s)=\frac{Tdx-Ixx*(yaw_{setpoint})*S^{2}}{Ixx*s^{2}+Kd*S+Kp+\frac{Ki}{s}}
\]

Equations for other axes will be similar only.

\section{Requirements:}
\begin{itemize}
	\item Designed controller should work for $Tdx=2*10^{-3}$,$Tdy=10^{-4}$,
	$Tdz=2*10^{-3}$Nm
	\item Steady state error<0.005deg
	\item Max overshoot=5\%
	\item Settling time(2\%) taken for design=500 secs
\end{itemize}
\section{Gain calculations:}

Formulas are derived will be same for all axes. Just moment of inertia
term will change

\subsection{Routh Criterion:}

\[
CE=IxxS^{3}+KdS^{2}+KpS+Ki=0
\]

This gives the following conditions.

\[
Ki>0
\]

\[
Kd>0
\]

\[
Ki<\frac{Kd*Kp}{Ixx}
\]

Since Ki has a destabilizing effect, we will take $Ki=Kd*Kp/Ixx$
for design

\subsection{Steady State Error Criterion:}

\[
E_{steady}=lim_{s->0}S*TF*1/S=0
\]

Because of integral term!!

\subsection{Settling time Criterion:}

Since Ki value is going to be small(Since Ixx is large). We can neglect
it here.

\[
2\zeta\omega_{n}=Kd/Ixx
\]

\[
\frac{4.4}{\zeta\omega_{n}}=Ts
\]

This gives us,\textbf{
	\[
	Kd=\frac{2*4.4*Ixx}{Ts}
	\]
}

\subsection{Max overshoot Criterion:}


\[
e^{-\frac{\pi\zeta}{\sqrt{1-\zeta^{2}}}}=\frac{5}{100}
\]

\[
Kp>\frac{Ixx*(4.4)^{2}}{\zeta^{2}Ts^{2}}
\]
\section{Gains calculated}
Thereby we got 3 formulas to calculate Kp, Ki and Kd gains. Kp given
by this inequality is very low. So Kp=1 is taken as a conservative
value. However KD and Ki are fixed as per the formulas found.
\begin{center}
	$\begin{array}{cccc}
	Gains & Kp & Ki & Kd\\
	X & 1 & 0.0088 & 32\\
	Y & 1 & 0.0088 & 28\\
	Z & 1 & 0.0088 & 21
	\end{array}$
	\par\end{center}
\section{Validation:}
\begin{figure}[H]
	\centering
	\includegraphics[width=1.0\linewidth]{simulation_at_constantTOrq_pryangles}
	\caption{}
	\label{Constant disturbance torque. Pitch, Roll and yaw angles}
\end{figure}
	Note: \\Product1 is Pitch, \\Product2 is roll \\ Product3 is yaw angle.
	Setpoint is Zero. \\
This simulation is done with constant disturbance torque on our Nonlinear simulink model to validate the gains.
\chapter{Momentum Dumping:}
\chapter{Simulation:}
\subsection{simulink Model:}
Since it is easier to visualise the flow of simulation in a graphical environment, simulink is used to simulate the system with controller.
It also made the system modular so that people can develop blocks separately and integrate them later.
\begin{figure}[H]
\centering
\includegraphics[scale=0.3]{Untitled1.png}
\caption{Simulink model}
\end{figure}
\subsection{Results:}

\end{document}